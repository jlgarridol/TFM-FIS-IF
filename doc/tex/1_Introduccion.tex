\capitulo{1}{Introducción}

La creciente facilidad para el acceso a Internet de alta velocidad por parte de la población del primer mundo ha permitido que los servicios de salud puedan llegar de manera telemática a la población alejada físicamente de los centros de salud y algunas tareas, como la rehabilitación de paciente de la enfermedad de Parkinson, se puede realizar con los modernos sistemas de videoconferencia.

Esta nueva forma de conexión entre terapeutas ocupacionales y sus pacientes puede ser apoyada con técnicas de inteligencia artificial. Estas pueden apoyar al terapeuta con su tarea rehabilitadora con información en tiempo real, incluso ayudar a la rehabilitación de muchos más pacientes a la vez. Para esta tarea, es necesaria tanto una determinada infraestructura técnica capaz de soportar y gestionar correctamente la carga de datos en tiempo real, como unos determinados modelos de inteligencia artificial capaces de procesar la información haciendo con un coste mínimo en tiempo y memoria.

Este trabajo de fin de máster está integrado en el proyecto \textit{Estudio de factibilidad y coste-efectividad del uso telemedicina con un equipo multidisciplinar para prevención de caídas en la enfermedad de Parkinson} del Ministerio de ciencia, innovación y universidades. Expediente \textbf{PI19/00670}. 

El objetivo de este trabajo consiste en la creación de la arquitectura de colas para el procesado de los vídeos de los pacientes en tiempo real en un entorno \textit{Big Data}. Además del aquí presentado existe otro TFM, desarrollado por el alumno José Miguel Ramírez Sanz, cuyo objetivo es la creación del algoritmo para el procesamiento del vídeo. Ambos trabajos dan el soporte informático al proyecto indicado.

\section{Material adjunto}

Junto a esta memoria se incluyen

\begin{itemize}
	\item \textbf{Anexos} donde se incluyen:
	\begin{itemize}
		\item Plan de proyecto
		\item Diseño del sistema
		\item Manual para el programador
		\item Manual para el usuario realizado junto a José Miguel Ramírez Sanz
	\end{itemize}
	\item \textit{Scripts} para el despliegue del sistema de colas para una instalación sobre máquinas \textit{Docker}.
\end{itemize}

Además se puede acceder a través de Internet al \href{https://github.com/jlgarridol/TFM-IF-FIS}{repositorio GitHub del proyecto}.