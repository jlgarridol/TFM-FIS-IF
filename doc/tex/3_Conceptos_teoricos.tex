\capitulo{3}{Conceptos teóricos}


En este capítulo se explicarán los conceptos teóricos subyacentes a todos los apartados del proyecto. Se afrontará en dos bloques, uno respecto a la teoría sobre la computación distribuida (\autoref{sec:tec}) y otro sobre el procesado de imágenes.

\section{Teoría de computación distribuida}\label{sec:tecdis}

\section{Teoría de procesado de imágenes}\label{sec:teccv}



\subsection{Definiciones}

\textbf{Imágen 24 bits}: codificación habitual de las imágenes a color. Cada pixel se define como la combinación de tres enteros sin signo de 8 bits. La codificación (o espacio de color) necesaria para la visualización es la combinación de las capas de color roja, verde y azul, conocida como RGB, aunque también la BGR usada por openCV. 

\textbf{HSV}: acrónimo del inglés de matiz, saturación y valor. Consiste en un modelo de color basado en los componentes del mismo y representa una transformación no lineal del espacio de color RGB.

\textbf{Matiz}: también conocido como tono, es el grado del ángulo respecto a la rueda de color~(TODO figura). Representa un color único y algunos ejemplos son el rojo (0\grado), amarillo (60\grado) o verde (120\grado).

\textbf{Saturación}: pureza del color tal que el valor mínimo es gris y el valor máximo es el tono puro. 