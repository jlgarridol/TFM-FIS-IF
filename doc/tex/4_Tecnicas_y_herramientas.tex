\capitulo{4}{Técnicas y herramientas}

\section{Gestión de flujo}	

Uno de los puntos más esenciales de este trabajo es recoger y dirigir los \textit{streams} de vídeo que se reciben. Por tanto, escoger una correcta aplicación para la gestión de este flujo de datos es esencial.

Dentro de la suite de \textit{Apache} existen varios componentes que se encargan de la gestión del flujo de datos. Con el objetivo de que el sistema fuese lo más robusto, y siguiendo las recomendaciones del estado del arte (TODO citas de esto) se combinarían las herramientas siguiente.

\begin{itemize}
	\item \textit{\textbf{Apache Kafka}}~\cite{noauthorapachenodate} es un proyecto de intermediación de mensajes que trabaja sobre el patrón publicación-suscripción funcionando como un sistema de transacciones distribuidas. Incorpora para la implementación de este patrón un sistema de colas para la distribución de mensajes. Aporta una API para el productor, el consumidor, el flujo y el conector y la conexión se realiza a través del protocolo de la capa de transporte \textit{TCP}.  
	\item \textit{\textbf{Apache Spark Streaming}}~\cite{noauthorsparknodate} es la extension sobre la API de \textit{Spark} para la creación de aplicaciones sobre flujos de datos. Es un consumidor nativo de \textit{Kafka}, \textit{Flume}, los sistemas de ficheros \textit{HDFS} y \textit{S3} entre otras herramientas. El funcionamiento interno es a través de crear pequeños lotes de datos para pasarlo al motor de \textit{Spark} y retornar los lotes procesados.
\end{itemize}


\section{Infraestructura de bajo nivel}

Otro apartado importante en el despliegue de la aplicación son las herramientas y técnicas a ser usadas para la producción. Para esto se utilizan:

\begin{itemize}
	\item \textit{\textbf{GNU/Linux}}, el sistema operativo por excelencia en el entorno de los servidores~\cite{noauthorred2018, zhang2000linux}.
	\item \textit{\textbf{Docker}}, un software de gestión de contenedores estandarizados, semejante a los entornos \textit{chroot} que facilita la virtualización de software en un entorno seguro y ligero. Sobre este motor se ejecutarán las aplicaciones del entorno de \textit{Apache}~\cite{juez2019docker}
\end{itemize}
