\apendice{Documentación técnica de programación}

\section{Introducción}

El objetivo de esta sección es que documentar a nivel técnico los componentes del proyecto para facilitar la extensión, modificación y uso por parte de los programadores.

\section{Estructura de directorios}

En la carpeta \texttt{doc} se encuentra la documentación del proyecto. En \texttt{src} los códigos fuentes para el proyecto.

Dentro de las fuentes están la carpeta \texttt{dockers} donde se incorporan tanto los \texttt{Dockerfile} como los \texttt{docker-compose.yml} que contienen la información de los contenedores que darán soporte a la aplicación. En \texttt{process} están las fuentes \textit{Python} para el emisor de fotogramas, el productor y el consumidor. En la carpeta \texttt{scripts} y en su subcarpetas \texttt{deploy} y \texttt{helpers} están todos los \textit{scripts} sobre \textit{Bash} necesarios para el despliegue de las imágenes \textit{docker} y las funciones que deben ejecutar, concretamente en \texttt{deploy} se encuentran los programas para la creación y cierre de imágenes y en \texttt{helpers} las órdenes sobre las aplicaciones que contienen las imágenes. Por último en \texttt{tools} se encuentran herramientas y \textit{notebooks} de \textit{jupyter} que se han desarrollado para desarrollar este trabajo.

\begin{figure}
	\dirtree{%
		.1 /.
		.2 doc.
		.3 img.
		.3 tex.
		.2 src.
		.3 dockers.
		.4 fishubu. 
		.5 base.
		.5 enviroment.
		.4 kafka.
		.4 spark.
		.5 base.
		.5 master.
		.5 worker.
		.3 process. 
		.4 testvideos.
		.3 scripts.
		.4 deploy.
		.4 helpers.
		.3 tools.
		.4 jupyter.
	}
	\caption{Árbol de directorios}
	\label{fig:dirtree}
\end{figure}

\section{Manual del programador}\label{sec:manualpro}

\subsection{\textit{Scripts Python} }
Se han creado tres \textit{scripts} de \textit{Python} para ofrecer el servicio del sistema de colas. Por el desarrollo que se ha llevado funcionan tanto dentro como fuera de una imagen \textit{Docker}. Si se fuesen a usar fuera a nivel local el uso sería el siguiente:

\subsubsection{\textit{emitter.py}}
\textit{Script} encargado en enviar a un servidor UDP fotogramas de vídeos. La ayuda:

\begin{lstlisting}[language=Bash]
Sintaxis:
	emitter.py --ip=localhost --port=12345 --file=video.webm
-----------------------------------------------------------
Parámetros de comunicación
	--ip=<IP de emisión> 
		(Por defecto: localhost)
	--port=<Puerto>
	--file=<Fuente de video>
	
Parámetros de gestión del flujo
    --resize=<Proporcion> 
    	(Por defecto: 1.0)
    -f <FPS> | --fps=<FPS> 
    	(Por defecto: 15) tasa de frames del vídeo a emitir
\end{lstlisting}

\subsubsection{\textit{producer.py}}
\textit{Script} encargado de la ingestión de los fotogramas a \textit{Kafka}. La ayuda:

\begin{lstlisting}[language=Bash]
Sintaxis
	producer.py --ip=localhost --port=12345 --topic=queue 
-----------------------------------------------------------
Parámetros de comunicación
	--ip=<IP de emisión> 
		(Por defecto: localhost)
	--port=<Puerto>
	--kafkahost=<Dirección de kafka> 
		(Por defecto: localhost:9092)
	--topic=<Topic de Kafka> 
		(Por defecto: video-stream-event)
\end{lstlisting}

\subsubsection{\textit{consumer.py}}
\textit{Script} encargado del procesado en paralelo y tiempo real del vídeo. La ayuda:

\begin{lstlisting}[language=Bash]
Sintaxis
	consumer.py --ip=localhost --port=12345 --topic=queue 
-----------------------------------------------------------
Parámetros de comunicación
	--output==<Carpeta de salida> 
		(Por defecto: output)
	--sparkhost=<Dirección de spark>
		(Por defecto: local)
	--kafkahost=<Dirección de kafka> 
		(Por defecto: localhost:9092)
	--topic=<Topic de Kafka> 
		(Por defecto: video-stream-event)
		
Parámetros de gestión del flujo
	-a -> Anonimizar rostros, por defecto pixalado
		-g <Factor> | --blur=<Factor>
			(Por defecto: 3) Anonimizar con blur
		-p <Factor> | --pixel=<Factor>
			(Por defecto: 15) Anonimizar con pixelado
	-b -> Auto ajustar brillo
	-c -> Auto ajustar contraste
	-f <FPS> | --fps=<FPS> 
		(Por defecto: 15) tasa de frames del vídeo a emitir
	--no-save -> No guardar los frames
\end{lstlisting}

\subsection{\textit{Script} de despliegue}

Para el despliegue de los servicios mediante \textit{Docker} se han creado cuatro \textit{scripts} encontrados en la carpeta \texttt{src/scripts/deploy} (en adelante \texttt{deploy}).

Estos códigos en \textit{Bash} son los siguientes:

\subsubsection{\textit{start-server}}
Se encarga de instanciar los diferentes servicios.

\begin{lstlisting}[language=Bash]
Sintaxis
	start-server <N. de CPU master> <N. de workers> <N. de CPU por worker> <Memoria por worker>
\end{lstlisting}

\subsubsection{\textit{stop-server}}
Detiene todos los servicios.

\begin{lstlisting}[language=Bash]
Sintaxis
	stop-server
\end{lstlisting}

\subsubsection{\textit{new-stream}}
Genera un nuevo flujo completo que se va a procesar. El funcionamiento es transparente. Recibe los parámetros para el \textit{emitter.py} y para el \textit{consumer.py}. Por seguridad es preferible solamente modificar los parámetros de gestión del flujo y no los de comunicación.

\begin{lstlisting}[language=Bash]
Sintaxis
	new-stream "Parámetros emitter.py" "Parámetros consumer.py" <Directorio de salida>
	# Es muy importante mantener las comillas
\end{lstlisting}

Devuelve el identificador del flujo. Es importante este valor para poder cerrarlo después.

\subsubsection{\textit{stop-stream}}
Detiene todos los procesos sobre un flujo concreto..

\begin{lstlisting}[language=Bash]
Sintaxis
	stop-stream <ID del flujo>
\end{lstlisting}

\section{Compilación, instalación y ejecución del proyecto}

Como se ha mencionado anteriormente en la carpeta \texttt{deploy} se encuentran los \textit{scripts} para instalar el proyecto y poder ejecutarlo.

Para que los \textit{scripts} se ejecuten correctamente es necesario que se ejecuten sobre un sistema operativo \textit{GNU/Linux} con el servicio de \textit{Docker} y la extensión \textit{Nvidia container toolkit}~\cite{toolkitnvidiadocker} instalados. Adicionalmente se necesitará que la máquina donde se vaya a ejecutar el consumidor tenga una tarjeta gráfica \textit{Nvidia} instalada con soporte para \textit{CUDA} 10.2 para que la integración con el proyecto del compañero José Miguel Ramírez Sans sea ejecutable. Es posible que sea necesario cambiar el fichero \texttt{Dockerfile} de la carpeta \texttt{dockers/fishubu/base} para que use los \textit{drivers} de la tarjeta gráfica del equipo sobre el que se lance la imagen \textit{docker}. Además si se quisiese cambiar el algoritmo, concretamente el fichero \texttt{extraOpt.py} sería necesario cambiar el \texttt{Dockerfile} de la carpeta \texttt{dockers/fishubu/consumer} para que sustituyese el algoritmo por el deseado.

Antes de lanzar los servicios es importante haber creado antes las imágenes maestras para los diferentes \textit{dockers}. Desde la carpeta \texttt{deploy}:

\begin{lstlisting}[language=Bash]
docker build -f ../../dockers/fishubu/base/Dockerfile -t fishubu-base:1.0.0 ../../
docker build -f ../../dockers/fishubu/enviroment/Dockerfile -t fishubu-env:1.0.0 ../../
docker build ../../dockers/spark/base -t spark-base-fis:2.4.5
docker build ../../dockers/spark/master -t spark-master-fis:2.4.5
docker build ../../dockers/spark/worker -t spark-worker-fis:2.4.5
\end{lstlisting}

El orden de ejecución de los \textit{scripts} es el siguiente:

Primero se ha de ejecutar \texttt{start-server} para que los servicios de \textit{Kafka} y \textit{Spark} estén activos y den soporte a los flujos que lo necesiten. Tras esto se podrá ejecutar \texttt{new-stream} recibiendo como parámetro la fuente de vídeo junto con la configuración deseada para el proceso (sección~\ref{sec:manualpro}). Este devuelve el identificador del flujo, será necesario para pedir el cierre del flujo.

Para parar la ejecución existen los \textit{scripts} \texttt{stop-stream} que recibe el identificador del flujo y \texttt{stop-server} que para los servicios de \textit{Kafka} y \textit{Spark}
