\apendice{Documentación técnica de programación}

\section{Introducción}

El objetivo de esta sección es documentar a nivel técnico los componentes del proyecto para facilitar la extensión, modificación y uso por parte de los programadores.

\section{Estructura de directorios}

En la carpeta \texttt{doc} se encuentra la documentación del proyecto. En \texttt{src} los códigos fuentes para el proyecto.

Dentro de los fuentes están la carpeta \texttt{dockers} donde se incorporan tanto los \texttt{Dockerfile} como los \texttt{docker-compose.yml} que contienen la información de los contenedores que darán soporte a la aplicación. En \texttt{process} están los fuentes \textit{Python} para el emisor de fotogramas, el productor y el consumidor. En la carpeta \texttt{scripts} y en su subcarpetas \texttt{deploy} y \texttt{helpers} están todos los \textit{scripts} sobre \textit{Bash} necesarios para el despliegue de las imágenes \textit{docker} y las funciones que deben ejecutar, concretamente en \texttt{deploy} se encuentran los programas para la creación y cierre de imágenes y en \texttt{helpers} las órdenes sobre las aplicaciones que contienen las imágenes. Por último en \texttt{tools} se encuentran herramientas y \textit{notebooks} de \textit{jupyter} que se han desarrollado para desarrollar este trabajo.

\begin{figure}
	\dirtree{%
		.1 /.
		.2 doc.
		.3 img.
		.3 tex.
		.2 src.
		.3 dockers.
		.4 fishubu. 
		.5 base.
		.5 enviroment.
		.4 kafka.
		.4 spark.
		.5 base.
		.5 master.
		.5 worker.
		.3 process. 
		.4 testvideos.
		.3 scripts.
		.4 deploy.
		.4 helpers.
		.3 tools.
		.4 jupyter.
	}
	\caption{Árbol de directorios}
	\label{fig:dirtree}
\end{figure}

\section{Manual del programador}\label{sec:manualpro}

\subsection{\textit{Scripts Python} }
Se han creado tres \textit{scripts} de \textit{Python} para ofrecer el servicio del sistema de colas. Por el desarrollo que se ha llevado, funcionan tanto dentro como fuera de una imagen \textit{Docker}. Si se fuesen a usar fuera a nivel local, el uso sería el siguiente:

\subsubsection{\textit{emitter.py}}
\textit{Script} encargado en enviar a un servidor UDP fotogramas de vídeos. La ayuda:

\begin{lstlisting}[language=Bash]
Sintaxis:
	emitter.py --ip=localhost --port=12345 --file=video.webm
-----------------------------------------------------------
Parámetros de comunicación
	--ip=<IP de emisión> 
		(Por defecto: localhost)
	--port=<Puerto>
	--file=<Fuente de video>
	
Parámetros de gestión del flujo
    --resize=<Proporcion> 
    	(Por defecto: 1.0)
    -f <FPS> | --fps=<FPS> 
    	(Por defecto: 15) tasa de frames del vídeo a emitir
\end{lstlisting}

\subsubsection{\textit{producer.py}}
\textit{Script} encargado de la ingestión de los fotogramas a \textit{Kafka}. La ayuda:

\begin{lstlisting}[language=Bash]
Sintaxis
	producer.py --ip=localhost --port=12345 --topic=queue 
-----------------------------------------------------------
Parámetros de comunicación
	--ip=<IP de emisión> 
		(Por defecto: localhost)
	--port=<Puerto>
	--kafkahost=<Dirección de kafka> 
		(Por defecto: localhost:9092)
	--topic=<Topic de Kafka> 
		(Por defecto: video-stream-event)
\end{lstlisting}

\subsubsection{\textit{consumer.py}}
\textit{Script} encargado del procesado en paralelo y tiempo real del vídeo. La ayuda:

\begin{lstlisting}[language=Bash]
Sintaxis
	consumer.py --ip=localhost --port=12345 --topic=queue 
-----------------------------------------------------------
Parámetros de comunicación
	--output==<Carpeta de salida> 
		(Por defecto: output)
	--sparkhost=<Dirección de spark>
		(Por defecto: local)
	--kafkahost=<Dirección de kafka> 
		(Por defecto: localhost:9092)
	--topic=<Topic de Kafka> 
		(Por defecto: video-stream-event)
		
Parámetros de gestión del flujo
	-a -> Anonimizar rostros, por defecto pixalado
		-g <Factor> | --blur=<Factor>
			(Por defecto: 3) Anonimizar con blur
		-p <Factor> | --pixel=<Factor>
			(Por defecto: 15) Anonimizar con pixelado
	-b -> Auto ajustar brillo
	-c -> Auto ajustar contraste
	-f <FPS> | --fps=<FPS> 
		(Por defecto: 15) tasa de frames del vídeo a emitir
	--no-save -> No guardar los frames
\end{lstlisting}

\subsection{\textit{Script} de despliegue}

Para el despliegue de los servicios mediante \textit{Docker} se han creado cuatro \textit{scripts} encontrados en la carpeta \texttt{src/scripts/deploy} (en adelante \texttt{deploy}).

Estos códigos en \textit{Bash} son los siguientes:

\subsubsection{\textit{start-server}}
Se encarga de instanciar los diferentes servicios.

\begin{lstlisting}[language=Bash]
Sintaxis
	start-server <Carpeta de salida> <N. de CPU master> <N. de workers> <N. de CPU por worker> <Memoria por worker>
\end{lstlisting}
\newpage
\subsubsection{\textit{stop-server}}
Detiene todos los servicios.

\begin{lstlisting}[language=Bash]
Sintaxis
	stop-server
\end{lstlisting}

\subsubsection{\textit{new-stream}}\label{sec:newstream}
Genera un nuevo flujo completo que se va a procesar. El funcionamiento es transparente. Recibe los parámetros para el \textit{emitter.py} y para el \textit{consumer.py}. Por seguridad es preferible solamente modificar los parámetros de gestión del flujo y no los de comunicación.

\begin{lstlisting}[language=Bash]
Sintaxis
	new-stream "Parámetros emitter.py" "Parámetros consumer.py"
	# Es muy importante mantener las comillas
\end{lstlisting}

Devuelve el identificador del flujo. Es importante este valor para poder cerrarlo después.

\subsubsection{\textit{stop-stream}}
Detiene todos los procesos sobre un flujo concreto.

\begin{lstlisting}[language=Bash]
Sintaxis
	stop-stream <ID del flujo>
\end{lstlisting}

\section{Compilación, instalación y ejecución del proyecto}

Como se ha mencionado anteriormente, en la carpeta \texttt{deploy} se encuentran los \textit{scripts} para instalar el proyecto y poder ejecutarlo.

Para que los \textit{scripts} se ejecuten correctamente es necesario que se ejecuten sobre un sistema operativo \textit{GNU/Linux} con el servicio de \textit{Docker} y la extensión \textit{Nvidia container toolkit}~\cite{toolkitnvidiadocker} instalados. Adicionalmente se necesitará que la máquina donde se vayan a ejecutar los \textit{workers} disponga una tarjeta gráfica \textit{Nvidia} instalada con soporte para \textit{CUDA} 10.2 (para que la integración con el proyecto del compañero José Miguel Ramírez Sanz sea ejecutable). Es posible que sea necesario cambiar el fichero \texttt{Dockerfile} de la carpeta \texttt{dockers/fishubu/base} para que use los \textit{drivers} de la tarjeta gráfica del equipo sobre el que se lance la imagen \textit{docker}. 

Antes de lanzar los servicios es importante haber creado previamente las imágenes maestras para los diferentes \textit{dockers}. Desde la carpeta \texttt{deploy}:

\begin{lstlisting}[language=Bash]
docker build -f ../../dockers/fishubu/base/Dockerfile -t fishubu-base:1.0.0 ../../
docker build -f ../../dockers/fishubu/enviroment/Dockerfile -t fishubu-env:1.0.0 ../../
docker build ../../dockers/spark/base -t spark-base-fis:2.4.5
docker build ../../dockers/spark/master -t spark-master-fis:2.4.5
docker build ../../dockers/spark/worker -t spark-worker-fis:2.4.5
\end{lstlisting}

El orden de ejecución de los \textit{scripts} es el siguiente:

\begin{enumerate}
	\item \textbf{Ejecutar \texttt{start-server}} para que los servicios de \textit{Kafka} y \textit{Spark} estén activos y den soporte a los flujos que lo necesiten.
	\item \textbf{Ejecutar \texttt{new-stream}} recibiendo como parámetros la configuración deseada para el emisor y para el consumidor (\autoref{sec:newstream}).  Este devuelve el identificador del flujo, será necesario para pedir el cierre del flujo.
\end{enumerate}

Para detener el flujo los pasos son los siguientes:
\begin{enumerate}
	\item \textbf{Ejecutar \texttt{stop-stream}} para cada flujo arrancado.
	\item \textbf{Ejecutar \texttt{stop-server}} y se detendrán todos los servicios.
\end{enumerate}

\section{Fallos y soluciones}

Los fallos que puedan ocurrir durante la ejecución de los \textit{scripts} dejan varios ficheros de log en la carpeta \texttt{/tmp/fishubulogs} así como en la carpeta \texttt{logs} de cada máquina \textit{docker} ejecutada. En caso de fallo del flujo es importante revisar los ficheros para conocer los posibles orígenes. Los fallos más comunes y sus correspondientes soluciones son:

\subsection{Fallo de memoria}
Si la caída del flujo deja una excepción del tipo <<\texttt{Caused by: java.io.EOFException}>> implica que los datos que debe procesar el flujo son mayores que los que caben en memoria.

Para solucionar esto es necesario dar más memoria a cada \textit{worker}. El valor recomendado es de 2 GiB por cada núcleo de \textit{worker}.

\subsection{Los fotogramas no se procesan}
Si los fotogramas no se procesan y observa la existencia del fichero de \textit{log} <<\textit{notprocesslog}>> en los \textit{workers}, significa que ha existido un error mientras se procesaba el fotograma. En el mismo \textit{log} se encuentra el origen del fallo y probablemente se deberá a que el fichero de \texttt{extraOpt.py} incluye información errónea o una ruta incorrecta.

Para solucionar esto se debe volver a cargar el fichero \texttt{extraOpt.py} y por tanto se han de volver a construir (orden \texttt{build}) las imágenes desde la \textit{fishubu-env}. En el caso de que el error persista, usar el flag \texttt{-{-}no-cache} a la hora de reconstruir las imágenes.

\subsection{Error de versión de \textit{Nvidia}}
Si a la hora de ejecutar el flujo, en el arranque ocurre una excepción del tipo <<\texttt{forward compatibility was attempted}>> indica que la versión instalada de \textit{Nvidia} sobre \textit{docker} no es compatible con la versión instalada en el equipo.

Para solucionarlo es necesario cambiar el fichero \texttt{Dockerfile} de la carpeta \texttt{src/dockers/fishubu/base} y cambiar la versión que se instala por la que se tiene en el equipo. Es necesario que al menos sea la versión~440.

\subsection{Error con la conexión a \textit{Nvidia}}
Si el flujo al iniciar da el error <<\texttt{could not select device driver}>> significa que la extensión de \textit{docker} para la compatibilidad con \textit{Nvidia} no está instalada o no se ha reiniciado el servicio de \textit{docker} desde que se instaló.

Se soluciona instalando la extensión de \textit{docker} <<\textit{docker-nvidia}>> y reiniciando el servicio.