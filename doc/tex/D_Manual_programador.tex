\apendice{Documentación técnica de programación}

\section{Introducción}

El objetivo de esta sección es que documentar a nivel técnico los componentes del proyecto para facilitar la extensión, modificación y uso por parte de los programadores.

\section{Estructura de directorios}

En la carpeta \texttt{doc} se encuentra la documentación del proyecto. En \texttt{src} los códigos fuentes para el proyecto.

Dentro de las fuentes están la carpeta \texttt{dockers} donde se incorporan tanto los \texttt{Dockerfile} como los \texttt{docker-compose.yml} que contienen la información de los contenedores que darán soporte a la aplicación. En \texttt{process} están las fuentes \textit{Python} para el emisor de fotogramas, el productor y el consumidor. En la carpeta \texttt{scripts} y en su subcarpetas \texttt{deploy} y \texttt{helpers} están todos los \textit{scripts} sobre \textit{Bash} necesarios para el despliegue de las imágenes \textit{docker} y las funciones que deben ejecutar, concretamente en \texttt{deploy} se encuentran los programas para la creación y cierre de imágenes y en \texttt{helpers} las órdenes sobre las aplicaciones que contienen las imágenes. Por último en \texttt{tools} se encuentran herramientas y \textit{notebooks} de \textit{jupyter} que se han desarrollado para desarrollar este trabajo.

\begin{figure}
	\dirtree{%
		.1 /.
		.2 doc.
		.3 img.
		.3 tex.
		.2 src.
		.3 dockers.
		.4 fishubu. 
		.4 kafka.
		.4 spark.
		.5 base.
		.5 master.
		.5 worker.
		.3 process. 
		.4 testvideos.
		.3 scripts.
		.4 deploy.
		.4 helpers.
		.3 tools.
		.4 jupyter.
	}
	\caption{Árbol de directorios}
	\label{fig:dirtree}
\end{figure}

\section{Manual del programador}\label{sec:manualpro}



\subsection{\textit{Scripts} de ejecución}

\subsection{\textit{Script} de despliegue}


\section{Compilación, instalación y ejecución del proyecto}

Como se ha mencionado anteriormente en la carpeta \texttt{deploy} se encuentran los \textit{scripts} para instalar el proyecto y poder ejecutarlo.

Para que los \textit{scripts} se ejecuten correctamente es necesario que se ejecuten sobre un sistema operativo \textit{GNU/Linux} con el servicio de \textit{Docker} y la extensión \textit{Nvidia container toolkit}~\cite{toolkitnvidiadocker} instalados. Adicionalmente se necesitará que la máquina donde se vaya a ejecutar el consumidor tenga una tarjeta gráfica \textit{Nvidia} instalada con soporte para \textit{CUDA} 10.2 para que la integración con el proyecto del compañero José Miguel Ramírez Sans sea ejecutable. Es posible que sea necesario cambiar el fichero \texttt{Dockerfile} de la carpeta \texttt{dockers/fishubu/base} para que use los \textit{drivers} de la tarjeta gráfica del equipo sobre el que se lance la imagen \textit{docker}. Además si se quisiese cambiar el algoritmo, concretamente el fichero \texttt{extraOpt.py} sería necesario cambiar el \texttt{Dockerfile} de la carpeta \texttt{dockers/fishubu/consumer} para que sustituyese el algoritmo por el deseado.

El orden de ejecución de los \textit{scripts} es el siguiente:

Primero se ha de ejecutar \texttt{start-server} para que los servicios de \textit{Kafka} y \textit{Spark} estén activos y den soporte a los flujos que lo necesiten. Tras esto se podrá ejecutar \texttt{new-stream} recibiendo como parámetro la fuente de vídeo junto con la configuración deseada para el proceso (sección~\ref{sec:manualpro}). Este devuelve el identificador del flujo, será necesario para pedir el cierre del flujo.

Para parar la ejecución existen los \textit{scripts} \texttt{stop-stream} que recibe el identificador del flujo y \texttt{stop-server} que para los servicios de \textit{Kafka} y \textit{Spark}
